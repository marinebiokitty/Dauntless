\documentclass[char]{airship}
\begin{document}
\updatemacro{\cThief}{
  \nickname{Lizzie}
  \mapnickinformal
  }
\updatemacro{\cCaptain}{
  \nickname{John}
  \mapnickinformal
  }

\name{\cCurie{}}
\bigquote{``Men are taught to apologize for their weaknesses, women for their strengths.''}{-- Lois Wyse}

You never should have married that bastard. What an idiot you were! He
took everything from you. Your precious daughter, who meant everything
to you, and your ground-breaking airship, your crowning achievement, he
took them both away from you and claimed them as his own.  But you'll
have your revenge soon, so very soon.

You were an impressionable young girl, working her way up through the
ranks of Her Majesty's Engineering Corps, he was a dashing airship
captain, serving in the Steam War... You were brought together by your
love of airships. He loved flying them, you loved designing them. His
talent in the latter is meager compared to yours. You were working on
a design for the the grandest airship of all time. You loved talking
to him about it. You let him think he was making useful suggestions,
but they were for the most part unworkable. However, you loved him and
didn't want to hurt his pride. You called your ship the {\em
Phoenix}. You hoped she could rise up from the remains of the Steam
War, a beautiful ship for peaceful times. Your design was too advanced
for the technology that was available then, but you were patient. You
knew you would make your dream come true, some day.

You marriage was all right at first, before and during the war. He was
off flying, you were at home, furthering your career, pursuing your
various scientific passions, and spending your free time with your
friends. You had so many different friends then! Friends from your
scientific societies, both the women's group and the mixed group,
friends from when you were at university, your friends from your
political groups, friends from the Corps. You missed your husband, but
you never lacked for excellent company. You hosted lively evenings of
debate at your home. You talked of the new theories of evolution and
all of the astounding new technological advances, debated the war, and
planned protests for women's suffrage. You can not help but smile when
thinking back on those exciting times.

When the war was over, you were overjoyed. No one likes living in a
country at war, and you would have your husband back. You looked
forward to hearing his bright, ringing voice debating new airship
designs in your parlor with your friends. It was not to be, though. He
came back from the war a changed man. His injuries caused him great
pain, but his personality was also transformed. He was less kind, less
forgiving. More harsh and rough. Less... civilized. You suppose that
is what war does to men... or perhaps that is what time does to
men. These days, you cannot say you are really a fan of the
pig-headed, Neanderthal-like half of the species. There are exceptions
of course, and you had mistakenly thought \cCaptain{} one of them.

\cCaptain{} no longer liked being around groups of people. You, of
course, held off on hosting any events at home when he first returned,
thinking it might take him a bit of time to readjust. Eventually you
cajoled him into allowing you to host a welcome home party for
him. You tried to appeal to his pride and made the party much more
patriotic than you actually felt. You didn't invite your most radical
friends, wanting to keep the party harmonious.

It was a disaster. \cCaptain{} started drinking before anyone arrived,
and never stopped. At least he wasn't being a loud, belligerent
drunk. He sat in the corner, being profoundly anti-social and drinking
continually. You tried to cut him off, but it was no use. You did your
best to keep up a good front, but everyone could feel the tension in
the air. Most of your guests left very early, but you couldn't blame
them. You didn't want to be there either. Eventually it was just you,
your husband, a few kind souls too polite to leave so early, and a
group of your artistic friends who were too busy drinking and singing
the praises of the Academie Julian in Paris to notice what anyone else
was doing. You were just as happy about the new coeducational art
school, and longed to join them in their joy, but you had to keep
playing hostess to the remaining guests.

Then \cBoddy{\intro} showed up. He was a friend of your husband's from
the war. \cCaptain{} suddenly lit up, jumped up from his chair,
and started chatting almost cheerfully. So, you thought, memories of that
war is all that brings him happiness? He didn't want to return to his
life in London with you, he wanted to linger in that damn war!

You made yourself a stiff drink and sat down with your friends,
smiling and nodding but not really listening to anything they
said. When the last of them staggered away you bade your husband and
\cBoddy{} good night and headed to your bed. You could hear them
laughing and singing till past sunrise.

You were already growing discontent with your marriage when you
discovered you were with child. You left your beloved career to
prepare for the birth of your child. \cCaptain{} became increasingly
reclusive and anti-social. You never had anyone over anymore. He did
not like you going out so much, but you needed to escape that
depressing house. You hoped the birth of your child would help your
marriage, that it would make your husband more like he used to
be. This happened to a small extent, in that your darling daughter
gave you and your husband something to focus on besides each other and
airships. You always had something to talk about, once you had little
Mary Elizabeth.

But a child can not really hold a marriage together. You no longer
loved him, and there was no denying it. Did you ever really love him,
or were you just infatuated? You were so young, so foolish...

You stayed with him for \cThief{}'s sake, because you had no money of
your own, and because it was easy. It was not that you were not still
fond of him, but it was a fondness born of years of familiarity, not
due to any redeeming qualities of his.

Ironically, it was \cThief{} that caused your ultimate falling
out. There was never the slightest doubt in your mind that you would
raise your daughter as a modern, liberated woman. Your husband knew
who you were when he married you; a woman, a scientist, and a
suffragette. However, he wanted to raise your daughter as a
traditional, stifled, repressed, miserable British woman. It makes you
feel both ill and angry when you see what that does to women. They
swallow so much pain, and it turns them into bitter, cruel
creatures. You do what you can to encourage them to change, to free
themselves, but some of them just don't want to! You have no use for
women like them, and there was absolutely no way you were going to let
that happen to your little girl. The fights you had over the issue
were impressive. \cThief{} was going to have all the opportunities you
could give her, and you didn't care what your ignorant, backward
husband had to say about it.

During all of your marriage you had continued your work with your
women's movement, but towards the end you had begun to focus your
efforts on reforming divorce laws, for obvious reasons. The laws were
ridiculous. In the case of divorce, men were given everything: the
children and all the assets. It wasn't until 1839 that, thanks to the
work of your friend Caroline Norton, women had even a chance of
getting any access to their children, and they still had to prove that
they were of ``unblemished character''.

Caroline was a good friend, a talented and famous writer, and a real
inspiration to you. She was the granddaughter of a famous playwright
and politician, but her family lived in greatly reduced
circumstances. She was forced into a loveless marriage by her widowed
mother for the financial security it would bring. Their marriage was a
disaster: he was a dull, useless, lazy Tory MP while she was a
brilliant writer and a Whig devoted to social reform. She refused to
pretend to love her husband, and because of it he beat her
severely. On several occasions she may have died if the servants had
restrained him. She tried to leave him several times, but she always
returned for the sake of her little boys.

Her husband, George Norton, made her use her friends to get him a
government position after he lost his position in Parliament. Lord
Melbourne, then the Home Secretary, got him an overpaid position for
Caroline's sake. Caroline and Lord Melbourne became great
friends. George was happy to ignore the (false) rumors of his wife
having an affair with Lord Melbourne; that is, until the greedy
bastard decided he could get more money out of Lord Melbourne by
threatening a scandal. It actually became a public scandal as Norton
and the Tories tried to blackmail Melbourne and force him out of his
position!

Norton's case against Melbourne, for ''alienating his wife's
affections'', was a complete joke. The jury did not even bother to
hear Melbourne's defense before finding in his favor.

However, because of the ridiculous laws of the time, George Norton was
still able to cut Caroline completely off from her children. Caroline
fought tirelessly to see her children again. Her pamphlet, {\em The
Natural Claim of a Mother to the Custody of her Children as affected
by the Common Law Rights of the Father}, was the first time a woman
openly challenged the discriminatory laws. Finally her efforts lead to
the very first piece of feminist legislation in England, which brought
some reforms to custody laws. Her husband sent the children off to
Scotland, beyond the hand of the law. It was not until their youngest
child died, having not seen his mother since he was two years old,
that George allowed the other boys to live with Caroline.

You and your friends continued the work of Caroline and all the other
wonderful women who came before you. Eventually women could not only
have access to their children in case of divorce, but they could
secure separations on grounds of cruelty and claim custody of their
children!

You wanted to wait until the laws were more in your favor, but when
Lizzie was four, you could wait no longer. You could not stand one
more day of that ridiculous situation, one more day watching your
wonderful little girl be influenced by the repugnant man who sired
her. You took matters into your own hands. You left. You packed up
yourself and \cThief{} and moved in with an artist friend of yours,
planning to stay there until you could find a permanent
arrangement. You took a job with the Engineering Corps. You had such
great hopes then, for your future with your daughter... but it was all
for naught, thanks to that hideous excuse for a human known as your
former husband.

He was exceptionally angry when he discovered your letter and your
absence. He apparently thought you ``had no right'' to do such a
thing. Had no right? Did he think he {\em owned} you? Many of your
friends reported terrifying visits from him, as he demanded to know
where you and \cThief{} were. Thankfully, he did not manage to find
you... then. 

Your divorce case was a profound miscarriage of justice. You
underestimated him. He used all his connections, all his old ``blokes
from the war'', especially that bastard, \cBoddy{}, and his deep
pockets. You were so naive! Your husband's barister called you an
unfit mother, accused you of having an affair with the friend you
were staying with (utter bollocks), accused you of introducing Lizzie
to inappropriate influences, to people of loose character, to
rabblerousers and anarchists and ``dangerous elements''. Complete
nonsense!

It didn't matter that none of it was true. The judge was bought and
sold before it began. His ruling was worse than you could have
imagined. Not only did he grant your ex-husband custody, he denied
you any right to even visit her. He also ruled that {\em your} work,
{\em your} airship designs, belonged entirely to your dirty husband!
He said it was ``ludicrous to imagine that such work could be the
product of a feminine mind, especially such a depraved one''. Those
words were etched in your heart that day, and they haunt you still.

You were devastated. You did the only thing that you could do - you
took your daughter and ran. There was clearly no justice for you in
England. You hid in France with Lizzie for three months. Every day,
you wondered if it would be the last day you spent together. You were
always afraid, but those days you spen together with Lizzie without
\cCaptain{} there were the best days of your life.

The last time you saw your little girl still haunts your dreams,
causing you to wake in a cold sweat, screaming until the tears take
over. Lizzie was screaming, calling out to you as the thugs tore her
from your arms. Her face was a mash of abject terror. You barely
noticed the pain of being kicked and thrown to the ground against the
pain of your heart breaking in twain.

A body shouldn't be able to continue functioning under so much pain,
but somehow you kept living. Well, your lungs continued breathing and
your blood continued circulating, but that is hardly living, is it?
You refused to talk. What use would it do, when your fate was sealed?
Not only did they lock you away, but you were banned from even {\em
writing} to Lizzie as long as she remained a minor. Does your little
girl even remember her Mama now? Does she remember how much you love
her?

You had nothing when you got out. You had been very publicly shamed
and the Engineering Corps wouldn't take you back. Caroline had died
and most of your friends had grown increasingly disenchanted with
England and her draconian ways and had emigrated to New York, or
Paris, or Munich. Besides, you despised everything about England at
that point. The only good thing England had was Lizzie, and you were
unable to see her. Your knew your old feminist friend Marie had moved
txo France, and you wrote to her. She was living outside Paris, and she
agreed to pay for your journey to France. When you arrived you found
she had changed. She was married, with two children and another on the
way. She'd put everything that had bound you together behind
her. She'd become bourgeois, and flat, and boring. Her husband clearly
resented you and wanted you out as soon as possible. You looked for a
way to support yourself, all the while wishing you were dead, but
unwilling to carry it out, to let the bastards win by driving you to
suicide.

You found a menial position in the engineering department of a French
university. You were generally disrespected because of your gender,
until you were transfered to work for another Professor,
\cHenri{\intro}. His father was English, but his mother was French and
he had been raised in France. He was quite interesting to you because
you were interesting to him. Amazingly, he seemed to have not even a
hint of the belief than men are superior to women. He was like a
breath of fresh air in a world clouded with misery. 
 
Henri and you became fast friends. Perhaps you could have been more
than that, if things had gone differently, but... He turned your
attention from airships, which had brought you nothing but pain, to
much smaller concerns. He was using technology to create, or simulate,
life! How many nights did you stay up past the sunrise, excitedly
modifying the latest prototype? You felt alive again since the first
time since you lost Lizzie, and the more free than you had been since
before \cCaptain{} came back from the war. Of course, nothing good
ever lasts, does it? Henri came down with consumption, and you were
terrified by the idea of his death. You moved into his home to care
for him. You tried to hide from your fear in work, desperate to move
beyond prototypes and create an actual life before Henri died. He
smiled at your fervor, your devotion to helping him complete his
work. You weren't willing to let him go, but you knew you were
watching his own life slip away as you tried to infuse that mysterious
force into your joint creation.

You moved both your beds into the workshop, abandoning the rest of the
chateau. Henri and your work were all that existed in your waking
world, while your dreams still belonged to Lizzie, sweet Lizzie.

Finally, you succeeded. A harsh winter was retreating in the face of
spring, the snows receding as the green began to take hold again. You
and Henri had created a beautiful little boy. Henri was bedridden by
that point, so you brought the child to his bedside. As his eyes
blinked for the first time and he looked up at Henri and you, you
asked Henri to christen him. ``William,'' he said, ``after my
father.'' The sight of his face full of such peace and happiness
brought you to tears.

``Don't cry, \me{\first},'' he said. ``I've lived to see my work
complete. It never could have happened without you. Yours is the
finest mind I have ever had the singular privilege to know. Always
remember that, no matter what any fool may say to you. You are a
string, brilliant woman, and you can go on without me. Teach our son
well, mon cherie.''

By that time your tears were cascading freely. Henri told you he was
tired and needed to sleep. You kissed his forehead and went to the
other side of the work room and quietly began instructing William. He
was so eager to learn everything you had to tell him. Any other day it
would have filled you with profound joy, but not this day.

Henri died the next day. You found yourself trying to explain
complicated human emotions to little William far earlier than you had
anticipated.

William was your life after that. He wanted to understand everything,
and you had so much to teach him! He could easily understand math, and
science, but the depths of human emotion were very confusing to
him. He was very naive, and you worried about what would happen to him
if he were exposed to the outside world. You loved him dearly, as much
as any ''real'' child.

Henri left everything to you. You never left the house. You had food
delivered, just as you had before Henri's death. You were so busy with
William that you forgot to deal with the legal issues of Henri's
death. Apparently there were significant taxes that you failed to
pay. All then notices were going to the university, and you never went
there.

You tried to hide William from the tax collectors. You kissed him and
told him you had to put him away for a little while to keep him safe,
told him you loved him, shut him down and hid him in the barn. They
dragged you away and seized everything. 

You spent months trying to find what happened to William, but it was
impossible. They claimed they had found no such thing anywhere on the
property, but he was gone when you came back. Two children lost to
you! At least you know that Lizzie is being taken care of, but
William! Who knows what happened to him? Is he being exploited? Is he
locked away in a box still? Has he been taken apart? Is anyone kind to
him? Your poor little boy!

Your thoughts have been turning more and more to your other lost child
lately, as her eighteenth birthday approaches. She will be an adult,
and no one will be able to keep you from seeing her. You are afraid,
though. It has been so long! Does she even remember you? Will she even
consent to see you, given all the horrible things she must have heard
about you all her life? He must have portrayed you as a hideous
monster who wanted to corrupt her and steal her away. You can only
hope she will give you the benefit of the doubt. What is she like? Has
she fallen into the crushing model of a proper Lady, or does she yearn
for more?  What are her favorite books, her favorite foods, what is
her favorite season? Does she look like you, or like her father? Does
she have your beautiful singing voice?

You're going to find out soon enough, you hope. Tonight you will be a
passenger on the {\em HMS Dauntless}, better known to you as {\em
Phoenix}. Yes, your pig of an ex-husband has finally built {\em your}
ship. Why have you been invited to the maiden voyage of Britain's new
crown jewel of the sky? Why, because you are now head of the Her
Majesty's Engineering Corps.

Yes, this is a very unexpected turn of events. You swore you would
never work for them again after the way they treated you when your
ex-husband destroyed your reputation, but now...

After you lost William, you turned back to airships, your first
passion. You returned to Henri's university and began working on airships
again. There were several other women in the department, liberated
women like yourself. You found the attitudes towards women in the
sciences were changing, at least in your department. Unfortunately you
still have to deal with chauvinistic, horrible men. You got into a
particularly severe shouting match with one bastard at a recent
conference, \cCid{\intro} you think his name was. It had been years
since anyone was so openly hostile and vulgar to you. It started out
as a technical disagreement, and eventually escalated to nearly a
shouting match before his friends dragged him off, muttering about
unstable, emotional women. He seemed to be personally offended by your
gender. While the field is certainly not integrated fully, you can't
have been the first female engineer he'd met. Every time you think the
world might be getting better, or your life might be getting better,
something like this happens to make you even more bitter! How you want
to show him and all the rest of them up! And you will, oh you will!

When her Majesty's Engineering Corps contacted you, you laughed out
loud. Go back to them? Were they insane? Even if it was the most
prestigious position you had ever been offered, you still hated
them. Then you read on and saw their newest ''achievement''... Your
husband had taken your plans and implemented them. Probably he just
handed them to a competent engineer to implement, because he certainly
couldn't do it himself. He was calling your ship the {\em HMS
Dauntless}. Well you were going to have none of that! You took the
position, to make sure you would be on the maiden voyage, but you made
certain... arrangements before you left France. The idea of your
ex-husband flying around in your ship, calling it his own, makes you
scream with rage. It is your ship, and you know her every weakness,
including how to extravagantly disable the ship without endangering
the passengers. Yes, yes, how could you leave such a flaw in the
plans?  Well, you were working to remove it, when your plans were so
unceremoniously seized by the courts. You had a friend arrange to hire
someone to sabotage the ship on her maiden flight. Your husband, and
the Royal Engineering Corps, will be humiliated. It will also provide
you a convenient excuse to resign from your position. Since coming on,
you have been publicly criticizing the {\em Dauntless}, saying it
will never make it to Paris, and announcing your intention to wear a
parachute the entire voyage. No one seems to remember that this was
{\em your} ship. Perhaps they all believed that hideous judge.

The Corps is quite unhappy with your behavior, but what do you care
for them? You won't be with them much longer! Tonight should be
glorious.  You can flaunt your success in front of your ex-husband,
watch him and the British Empire be publicly humiliated
and... hopefully... be reunited with your daughter. You'll have to
approach the situation delicately, of course, and brace yourself in
case she is hostile to you, but you must find a way to win her
over. She's of an age to attract suitors now, and you can only hope
that she won't repeat the mistakes of her mother and fall for some
fiend.  If only you could find William as well...


\begin{itemz}[Goals]
  \item Reunite with your unjustly-taken daughter.
  \item Find William.
  \item Humiliate your ex-husband, and England, by sabotaging the ship.
  \item Make sure Lizzie's suitors are upstanding young men.
  \item Show the world what a woman can do!
\end{itemz}



%%%%%
%% List contacts, using \contact{<char macro>}
\begin{contacts}
  \contact{\cCaptain{}} Your subhuman ex-husband.
  \contact{\cVicar{}} Your husband's brother, a minister.
  \contact{\cBoddy{}} The owner of the {\em Dauntless}, and a horrible human being.
%  \contact{\cHenri{}} A gracious, now deceased, friend.
%%  \contact{\cRobot{}} Your wonderful creation.
%  \contact{\cNinja{}} One of the men you hired to disrupt the ship.
\end{contacts}


%%%%%
%% \starttag{<tag>} <elements> \endtag 
%% Valid <tag> values are blues, greens, abils, combat, mems, items,
%% whites, notebooks, cash, signs, ids.  These each correspond to a
%% type of macro defined in Lists/.
%%
%% By using \starttag, you can give this character <elements> of the
%% type corresponding to <tag>.
%%
%% Multiple uses of the same <tag> will simply add together.
\starttag{mems}
  \memfold{Open if you see badge \#\cCid{\MYnumber}}{Oh! -  it's that terrible man who was so rude to you at that conference! How unpleasant! You'll show him up!}
  \memfold{Open if you see badge \#\cRobot{\MYnumber}}{It's William\ldots{}}

  \memfold{Open at game start.}{A series of explosions rock the ship, and you are thrown to your knees.  Why are those sabouteurs you hired starting this early?}
  \memfold{Open if you see badge \#235}{Could this beautiful young woman be your Lizzie, all grown up?}
  \memfold{Open if you talk with badge \#235}{Lizzie's voice sounds off. Not at all like it used to be. Has she been neglecting her singing lessons?}
\endtag
\end{document}
