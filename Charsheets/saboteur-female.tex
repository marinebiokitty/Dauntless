%%%%%
%%
%% Character sheets live in this directory.  This file doubles as a
%% latex'able example charsheet.
%%
%% Character macros (in ../Lists/char-LIST.tex, presumably) each
%% have a file that lives here.  The argument to \name{...} probably
%% should be the macro for the given character, which will generate
%% the charsheet's name (and print out lists of the characters stuff
%% at the end) as specified in char-LIST.tex.  However, you can also
%% just use \name{Some Text} if you want.
%%
%%%%%

\documentclass[char]{airship}
\begin{document}

\name{\cSaboteur{}}


\bigquote{``The only people for me are the mad ones, the ones who are
mad to talk, mad to live, mad to be saved, desirous of everthing at
the same time, the ones who never yawn or say a commonplace thing, but
burn, burn, burn like faboulous yellow roman candles exploding like
spiders across the stars and in the middle you can see the blue
centerlight pop and everybody goes, 'Aww.'''}{-- John Keroac}

They accused you of being a dreamer.  That you wanted, wished for
things that were impossible to achieve, because of human nature.  The
accursed ``human condidtion'' that would be cited at you time and time
again as reasons why your dream for a better world for mankind was
just that: a dream.

It wasn't exactly the slums of Paris that you grew up in, but it was
close enough that you saw human suffering every single day.  The
working class, mostly.  Good, honest people who worked their hands to
the bone day in and day out in the soot-stained factories that fed the
economic and political ambitions of the French empire.  People who
didn't even get a taste of the wealth and glory their blood, sweat,
and tears ultimately earned.

Despite all of this, your parents did the best they could for you,
including spending what precious little extra money they had on
second-hand books that you practically devoured.  Read nearly
everything you could get your hands on.  Homer, Virgil, even some
Voltaire and Nietzche, you managed to get yourself an education beyond
the state-regulated one on your own.  And what you read, you
remembered.  Not to say that it was easy.  Of course not.  Who ever
heard of a classically-educated girl living in the working-class
neighborhoods of Paris, reading philosophy and mythology while drunken
louts carried on right outside her window?

Nevertheless, you survived, and despite all odds, in the face of
adversary, you thrived.  You saw that human suffering for the entirety
of your childhood, and as a young woman ready to make her mark on the
world, you swore that you would fix all that was wrong with the world.
You had the intelligence, if not the money, and the drive to do it,
and nothing was going to stop you.

Idealistic, that's what you were.  For all your intelligence, you
initially underestimated the value of deep pockets.  No one would
listen to a fresh-faced young woman with all her revolutionary ideas.  Not only because of your gender, although you're sure that had some part in it, but your ideals were a threat to the social order, and as such, a threat to those in power.  You were warned by your parents (bless their souls) and friends, that despite how much they supported you in whatever you did, you did have to eat.  That and getting thrown into prison for your beliefs would not be good for your parents in any way. 
Those at the top paid enormous sums to keep the status quo the way it
was.  They dealt harshly with those whom they saw as a threat to their power.  The only way to change the system, you saw, was to enter it and exploit it.

You proved yourself quite adept at that trick.  You started
manipulating the system with the most powerful tool you had: your
mind.  All those years of reading paid off.  You adapted all the
various things you had read into your own works that you wrote, essays and novellas, mostly, that subliminally promoted the ideals you had gleaned to the people who read them.  These were fairly well-recieved, to your surprise.  Your crowning achievement as a writer, in your opinion, is your recent and first novel, ``Ragnarok'' (a name which amuses you still).  It is you, your beliefs and ideals, poured out into written form and given to the world in a way that you felt that you could get your message across to the them.  It was risky; the public might not be ready for that and you would most definately be branded as a radical and persecuted as such.  But you took that risk and published your book.  And the public loved it.  Seemingly overnight, you became a successful author.  The books you wrote went
flying off the shelves almost as fast as the publisher could print
them.

It just didn't seem right, though.  People bought and read your works,
but with an air of ``Oh, how nice.  What a wonderful story,'' and then
went about their lives as if nothing ever happened.  It grated on your
nerves.  You had poured your soul into these words, hoping someone
would see the deeper message in them and agree with you, that enough
of them would do so to revolutionize the world.  But people were too
blind, too complacent.  As much as it pained you to admit, they
wouldn't change unless they were forced to.  They simply didn't see a
way out of their existences as it stood.

One day, while you happened to be walking through a bookseller's, you
happened upon a new book, freshly translated from German, called ``Das
Kapital'' by a man named Karl Marx.  On a whim, you picked it up.
That decision changed your life.  Marx was on to something.  Those in
power would continue using the system to their advantage until it
couldn't take it anymore, and the working class people, your people,
rose up and changed the world.  Here was a man who had crystalized all
of your thoughts and ideals into a single text; here was a man who
agreed with you, and had an idea of how to fix everything that was
wrong with the world.  Best of all, rahter than fighting against human
nature, it would be human nature that caused the revolution.

But, as you know all too well, people were a bit too lazy, a bit too
complacent with their place.  They couldn't, or at least wouldn't, see
the big picture or the long term.  Any kind of revolution by the
prolitariat would be a long time in coming.  But if you gave them a
reason, pointed out what was wrong with the decadence of the upper
class in a way that no one would see coming nor forget, it could
galvanize the people to start the revolution.  After reading Marx's
work, you met up with like-minded people who also realized the
importance of action to jump-start the prolitariat.

A few weeks ago, you recieved a message from the group.  Apparently,
the British are eager to show off their latest example of decadence
and extravagance: the brand-new airship {\em HMS Dauntless} is making it's
maiden voyage from London to Paris on New Year's Eve.  Someone has
hired your group to sabotoge the voyage.  There are supposed to be
fairly high-ranking members of the British aristocracy onboard, no
doubt enjoying lazing around on a vessel the working class broke their
backs over to build.  What better way to send a message that such
decadence should not be tolerated than to blow the ship out of the
sky, in front of thousands over the Parisian skyline?

The group picked you to go.  As a fairly well-known writer, it was
hardly difficult to get yourself on the guest list.  So here you are,
in London, armed with enough explosives and tools to do the job.  Of
course, no one has any idea that you are anything but a prolific
author, and it's definately in your best interest for it to stay that
way. 

Besides, you received this mysterious letter from the Mayor of the 
Dutch Cape Colony, who wants you to utilize your contacts in France and
secure some sort of capital for his diamond trade. You're not sure how they found
you, or why they came to you in particular - perhaps the Mayor is confusing you with some French loyalist espionage group.
All for the better - you'd love to secure the wealth that shares in an up-and-coming
diamond company would yield. You're willing to do whatever it takes to ensure that
you talk to Blackwell before any thrice-damned \textit{noble} does.

Oh, yes. You intend to give the people of Paris and the world a fireworks
show they'll never forget.

%%%%%
%% The itemz environment is a list environment similar to itemize.
%% The typesetting is very tight, and matches that used by the lists
%% at the end of character sheets.  It takes an optional argument that
%% acts as a title for the list.  The enum environment is a similar
%% variation of the enumerate environment, and the desc environment is
%% similar to description.
\begin{itemz}[Goals]
  \item Sabotage the ship and escape before it crashes.
  \item Retrieve the shares of the De Beers Company from Blackwell.
  \item Assassinate the leaders of the Conservative and Labour Parties, as a last resort.
\end{itemz}

\begin{itemz}[Notes]
  \item You are undercover. There will probably be oppisition to your mission, so try to keep hidden.
  \item Murder is messy. You want the nobility to die, and you'd much prefer if it were when the airship crashes into the hard ground, but you won't hesitate to doing it by hand.
  \item What you are doing is highly sketchy.  If questioned about what you are doing in a certain part of the ship, lie.  Make something up.  You're an author; this should be trivial.
\end{itemz}


%%%%%
%% List contacts, using \contact{<char macro>}
\begin{contacts}
  \contact{\cCaptain{}} The captain of the ship, and the only person here you're even remotely familiar with.
  \contact{\cBoddy{}} The man you're to talk to about the diamond company. You can recognize him from the picture you saw in an English newspaper.
\end{contacts}


%%%%%
%% \starttag{<tag>} <elements> \endtag 
%% Valid <tag> values are blues, greens, abils, combat, mems, items,
%% whites, notebooks, cash, signs, ids.  These each correspond to a
%% type of macro defined in Lists/.
%%
%% By using \starttag, you can give this character <elements> of the
%% type corresponding to <tag>.
%%
%% Multiple uses of the same <tag> will simply add together.
\starttag{mems}
%%  \mTest{}
\memfold{Open at the start of game.}{A series of explosions rock the ship, and you fall to your knees. Damnit! You didn't plant that! Maybe somebody else here has started to dismantle the ship before you did.}
\endtag

\starttag{abils}
\ability{Debate}{If you spend three minutes discussing Marx, ``Ragnarok'', the struggle of the working class, or something similarly appropriate, you may play this ability on anyone you were discussing it with for the whole duration. It will help recruit them to your purposes. Only people with a $\chi$ stat are recruitable.}{This conversation piques your curiosity. Subtract one from your $\chi$ stat. If you do not have a $\chi$ stat, tell me so.}
\ability{Advanced Sabotage}{You have one bomb you can plant on the ship. To do this, take the physrep (a sign, a timer, and a deck of cards) out of the ``Not in game'' box in 34-3 and tape it on the floor in plain sight in the 34-3 lobby. This is the only location where the detonation will severely damage the ship. Do not start the timer. If you plant the bomb later than half an hour before the end of game, the ship will land before the bomb goes off.}{...} 
%%
\endtag

\end{document}
