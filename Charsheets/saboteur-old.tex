%%%%%
%%
%% Character sheets live in this directory.  This file doubles as a
%% latex'able example charsheet.
%%
%% Character macros (in ../Lists/char-LIST.tex, presumably) each
%% have a file that lives here.  The argument to \name{...} probably
%% should be the macro for the given character, which will generate
%% the charsheet's name (and print out lists of the characters stuff
%% at the end) as specified in char-LIST.tex.  However, you can also
%% just use \name{Some Text} if you want.
%%
%%%%%

\documentclass[char]{airship}
\begin{document}

\name{\cSaboteur{}}


\bigquote{``Use this macro for large quotes of prose and such.  It
justifies everything like a paragraph, except with no
indentation.''}{-- The Author}

\cenquote{``This macro is good\\ For shorter quotes\\ Or things like
song lyrics:\\ It centers.''}{-- The Author}


\TODO{This is a test character sheet.}


- focused, calm, political
                   - kill noble or blow up airship (both is best!)
                   - Parisian writer
                   - Has Parachute


This is a test character sheet.  Your name is \me{}.  Your friends
call you \me{\informal}, while others call you \me{\formal}.

This is some text.  This is some text.  This is some text.  This is
some text.  This is some text.  This is some text.  This is some text.
This is some text.  This is some text.  This is some text.

Your friend, \cNPC{\informal}, is a nice person, if a bit clueless.


%%%%%
%% The itemz environment is a list environment similar to itemize.
%% The typesetting is very tight, and matches that used by the lists
%% at the end of character sheets.  It takes an optional argument that
%% acts as a title for the list.  The enum environment is a similar
%% variation of the enumerate environment, and the desc environment is
%% similar to description.
\begin{itemz}[Goals]
  \item Things to do
  \item Governments to topple
  \item Worlds to dominate
\end{itemz}

\begin{itemz}[Notes]
  \item You were born in London.
  \item You went to MIT, and never left.
\end{itemz}


%%%%%
%% List contacts, using \contact{<char macro>}
\begin{contacts}
  \contact{\cNPC{}} A test contact.
\end{contacts}


%%%%%
%% \starttag{<tag>} <elements> \endtag 
%% Valid <tag> values are blues, greens, abils, combat, mems, items,
%% whites, notebooks, cash, signs, ids.  These each correspond to a
%% type of macro defined in Lists/.
%%
%% By using \starttag, you can give this character <elements> of the
%% type corresponding to <tag>.
%%
%% Multiple uses of the same <tag> will simply add together.
\starttag{mems}
  \mTest{}
  \memfold{``Rosebud''}{Rosebud!  That was the name of... the name
  of... darn, you forget.}
\endtag

\starttag{abils}
  \ability{Amazing Powers}{You can do strange and amazing things.}{I
  do something strange and amazing.}
\endtag


\end{document}
