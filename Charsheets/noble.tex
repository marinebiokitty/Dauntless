\documentclass[char]{airship}
\begin{document}

\name{\cNoble{}}

\bigquote{``England reigns! England reigns!''}{-- Edwin Drood}
    
It's surprising how the state of one's life can mimic the state of
one's country. In your case, that's a very fine thing. England
prospers, albeit clumsily, and you along with it. For you are Lord
David Lloyd Major, head of the powerful Labour party in the House of
Lords. You firmly believe that England is at the forefront of an
industrial revolution, and that your great nation's future and
continues dominance will be determined not by your soldiers and
nobility, but by an endless army of engineers and scientists. A
technically savvy corps of fine young men who will remake the world
twice over in steam-driven glory!

But, like you said, clumsy. Your opponents in the Conservative part of
Parliament have different ideas, for example, and the lesser nations
of the world attempt a bite at the jugular of the Lion of Europe. At
least the Conservatives only want to bring England back a hundred
years or so. It sometimes seems like the desperate and vicious nations
of the Earth would rather bring the British Empire back into the Stone
Age. Oh, well. ``The sun never sets on the British Empire.'' That's
probably your favorite phrase, for two reasons. First, the dominions
and holdings of this great country are so large that the sun is
literally always shining on some part of it. But it's also a very
optimistic statement. The sun always shines - the door of opportunity
is always open. There is always time for progress!

In fact, a certain French author, \cSaboteur{} has some interesting ideas about progress. He's written a book, you see, called ``Ragnarok'', that depicts some alternate world wherein the working class overthrow their ``enslavers'' in the nobility. A rather sensationalistic book, but for some reason it's been the all the rage among Labour Party ranks. Although its ideas are rather idealistic and sometimes downright strange, the writing is wonderful. You wouldn't mention it, but \cSaboteur{} seems to be on the guest list for this maiden voyage of the {\em HMS Dauntless}, so you're looking forward to chatting with him, as you're sure much of the ship is, whatever their political bend.

But sometimes your two parties work together for some end. You've
recently been speaking very plainly with Lord Cornelius Blackwell, the
leader of the opposition Conservatives in the House of Lords about the
threat of France, and the two of you, in a dark hour, came up with a
scheme. France is catching up much too quickly to your country's
mechanical advantage, and needs to be put down. As you debated what
could be done, it occurred to both of you simultaneously to flood the
French countryside with opium. Lord Blackwell indicated that he knew
someone who could deal such a large quantity, which you would then
use your connections with the laborers of this nation to import into
France. All of this is expensive, of course, and while the members of
the Labour Party do not know what they are paying for, you're sure
they'd be happy if you did all you could to keep France down. So
you're here with an envelope of several thousand pounds. When you meet
up with Blackwell, he should point you in the direction of the dealer,
or just do the exchange for you.

And you're still not sure what to make sure of this letter that you
received that speaks of the diamond trade on the Cape. You've researched
Mayor Rhodes, and his credentials are solid - Oxford schooling, as he writes,
moderately important politician around the Cape. Still, your lieutenants
in the party assure you that you have the money and the forces to secure the
shares. In fact, you could work it into your politics - send troops to Cape
Colony now, before the Dutch lose control of the area! You'll have to think
more about it when you actually speak with Blackwell.  If there are others who 
are also looking to make the deal, you intend to be the first to talk Blackwell
into giving you the shares. Even if things turn ugly, you won't be phased - you 
have bigger problems, particularly at home.

Your wife... God, how you regret marrying that harpy. She has spent
all the years of your marriage trying to suck away any joy you might
ever feel. You married her for her money, and her name, and because
your father told you to. You expected a normal marriage. She would run
the household and raise the children, you would deal with all outside
concerns. She seemed very concerned with the correct ordering of
things, with proper manners and behavior and religion. That was fine
by you! You thought she would be {\em quiet} and {\em obedient} to her
husband's will. For one who has so much to say about the proper role
of women and men, she certainly does not practice it herself.

\cJack{} finds fault with absolutely everything, especially anything
you do. You have many times pondered slicing off your own ears to get
away from her horrible shrill voice. You could forgive her irritating
personality and constant complaints if it were not for what happened
to your children, your dear, dear children.

\cJack{} gave you three children, Augustine, Thomas Aquinas and
Abigail. After she named your two sons after religious figures you
demanded the right to name your daughter after your late mother. You
adored your children, and doted on them. They brought life and light
and joy into your home, things that were sorely lacking when it was
just you and \cJack{}. They were clever, kind and happy children. As
was traditional, you left their religious education to their
mother. Truth be told, you yourself do not have much in the way of
religious inclination. To all outward appearances, your children were
being brought up in the normal way. It was not until they began
approaching adulthood that things became... difficult.

Augustine, your eldest, was a normal boy, but he listened to his
mother too much. She was convinced her was drenched in sin, and he
believed her. By the time you realized how much she had warped his
mind, it was too late. They were always praying together, consulting
with the vicar, agonizing over the state of his immortal soul. You
were more concerned about his growing into a normal boy. You tried to
give him some balance in his life, to get him to care about the things
normal boys care about, to get out of the house and play with the
other children! Your wife accused you of ''interfering'' in women's
work and trying to corrupt your own son! Convenient that she only
brings up women's work and men's work when it suits her current needs.

You relented some, hoping Augustine would eventually outgrow this
phase, and instead made sure the same thing would not happen to your
other son, Thomas Aquinas, or Tom as everyone called him. Well, as
everyone called him when not within earshot of his mother. She said
Tom was a vulgar name and insisted he be called Thomas Aquinas. Tom
tried to be a dutiful son, but as time went on, he had as little use
for his mother's mad rantings as you did.

Augustine became increasingly distraught under his mother's
influence. His siblings also tried to help him, but no one knew what
to do. The Vicar, \cVicar{\intro}, came to your home frequently to
consult with your wife and son. You were all completely shocked and
devastated when he took his own life. \cJack{} was horrified that one
of her children could commit such a grievous sin and refused to allow
his name to be spoken in her presence after his funeral. You hoped
that some of her response was due to guilt over driving your son, your
heir, your darling boy, to such an extreme act, but you fear that she
really feels no guilt, that she thinks she was helping the boy, not
destroying him.

Tom was especially devastated by his brother's death, and openly
blamed his mother. While you felt the same way, you tried to keep
things civil at home. Tom began rebelling more and more. He saw no
reason to pretend to go along with his mother, and you had trouble
blaming him. She took to punishing him severely, not at all in line
with the actual depth of his offense. When you tried to intercede on
his behalf, she transformed into the most hideous creature you have
ever encountered. Her entire face turned scarlet, and you swear her
hair stood on end. Little Abby cried for a full day afterwards, and
she was not even in the same wing of the house. You consented to
sending Tom to a reform school in hopes that he would be happier away
from his mother. He was back within a few months due to his
behavior. You did not know what to do for you son. He told you he
understood, that he knew you were just as trapped as he was. 

By the time Tom was sixteen, the situation was unbearable for
him. After his mother locked him a closet for two days while you were
away, Tom came to you and told you he could no longer live under her
roof. If you could not provide a good life for your son at home, you
were at least going to make sure he could have a good one outside. Tom
wanted to explore, to get out of England. You used some connections
and set him with a good position in South Africa. You have been
sending him money periodically from an account your wife doesn't know
about, but he tells you he is doing so well now he no longer needs
it. You keep in as frequent contact with him as you can, sending mail
from your office instead of your home. You miss your son dearly, but
are glad he has found happiness. He tells you he has been courting a
young Englishwoman, living in Cape Town with her parents. It hurts you
to think of not being at your son's wedding, of not being able to hold
your grandchildren. However, your wife is still a problem.

Just like after Augustine's death, she refused to speak of Tom, though
you could see she was hurting. You wanted to pity her, but it was her
behavior that drove him away! The loss of both of your sons left you
quite melancholy. You took to drinking to much, and locking yourself
in your study, away from your hideous wife.  With Tom gone, she
focused all her attention on poor Abby.

Surrounded by brothers, a mother who proclaimed all the virtues of
womanhood while practicing none of them, no other appropriate female
role models, it's no surprise the girl turned out the way she did. She
was always too interested in masculine pursuits. Wanting to get out of
the house and play away from her mother was completely understandable,
but she always wanted to play with machines, to build things, to
tinker. You hoped she would grow out of it, but \cJack{} acted like
her hobby was Judgement Day come at last. Your wife tried to repress
Abby as much as your other children, and like Tom, she eventually ran
away. You wish she had told you what she was going to do, you could
have helped her. One morning she was just gone, with only a note left behind.

\begin{quote}
Dear Mother,

Goodbye. I am leaving to escape your madness. Don't worry about me, I
can support myself. Don't blame Father, it is your fault I've been
driven to this.

Abby
\end{quote} 

You miss all of your children. You had not realized how much Abby
improved your life until she was gone. You have searched for her, but
to no avail. If only she would write to you!

With all of the children gone, you saw little reason to stay at
home. Your wife suddenly stopped talking for a year, but you barely
noticed. You still ate dinner at home, because she was a complete tart
if you didn't. One might think it would be hard to be such a tart while
refusing to speak, but only if one didn't know your wife.

You took to gambling. It was a distraction from your horrible home
life. You hate your wife, and you would leave her, but the majority of
your family's wealth comes from her brother, and if you left her, you
would lose that. She is not worth becoming destitute over. 

One night, a lovely young woman came into the club you spend most of
your time at. She flirted with you. It had been so many years, you had
forgotten what it was like to have a beautiful woman pay attention to
you. You wound up in bed with her. You didn't care that she was a
whore. You hadn't had sex in so many years, you'd forgotten how
wonderful it was! Suddenly, there was a new spring to your step, a new
light in your eye. Life became suddenly tolerable again! You have
enjoyed the company of many lovely young ladies in the past year. They
get paid well, and you get your pleasure. It is a fine arrangement.

Around the same time, your wife finally grew tired of her silence and
began talking again. You began acting like a normal family again. It
was... not so much pleasant as neutral. So you try, now and again, to
upkeep the facade that your marriage is neutral. You had your wife
bring aboard a most famous piece of jewelry, the Neptune's Tear, to
wear at the on-flight ceremony that is scheduled to occur. It's
something that was given to you by an African prince in your travels
as head of Labour, a unique and priceless artifact, and one that you
especially treasure. Its successful exhibition will add that much more
bargaining power when you indulge your more primal instincts.

You have spent quite a bit of time with the ladies of the night, but
they are not your only companions. While there are certain advantages
to the company of experienced ladies, you must admit there is great
pleasure in having your way with young ladies who have have not been
touched by other men. 

While you still miss your children terribly, your life is all in all
looking up, except for an unfortunate incident last night at one of
the many parties associated with the launch of the {\em
Dauntless}. \cDealer{\intro}, one of the other passengers on the ship,
introduced you to a lovely lady, \cNPCWhore{\intro}. You had a good
time together, and one thing lead to another, and you seem to have
left your wedding ring in a most intimate location. You have to find
her as soon as possible and get your ring back, before your wife
notices.

This ship is teeming with attractive women. There is \cNPCWhore{}, who
you encountered last night, and \cWhore{\intro}, one of the women of
negotiable virtue you have been seeing. Now, \cThief{\intro}, the
Captain's daughter{\ldots} oh you would love to get your hand on (or
in) her. She portrays herself as a prim, proper young virginal
daughter, but you know that girls like that can be the most fun once
you show how. You see a real spark in her eyes, and you could swear
you have seen her out on the streets, streets where no proper young
girl should be. Perhaps you are imagining things, but if you are right
about her, oh what fun that would be!

\begin{itemz}[Goals]
  \item Deliver the money to Cornelius and his opium supplier.
  \item Find those Cayman Island charters.
  \item Get your ring back before your wife notices.
  \item Make sure your wife displays the Neptune's Tear at the Dedication.
\end{itemz}

\begin{itemz}[Notes]
  \item You are a very important person in politics. Act like it.
\end{itemz}


%%%%%
%% List contacts, using \contact{<char macro>}
\begin{contacts}
  \contact{\cJack{}} Your harpy of a wife.
  \contact{\cBoddy{}} Your opposition in the House, but working together with you today.
  \contact{\cNPCWhore{}} A lady you've had a daliance with. You fear you may have left your wedding ring{\ldots} with her.
  \contact{\cWhore{}} Yet another of the ladies of the night whose services you have been making use of. Everyone seems to think she is governess or something odd{\ldots}
  \contact{\cThief{}} The captain's daughter. Very, very tempting
  \contact{\cVicar{}} Your family's minister, and the brother of the ship's captain. He is recently returned from America.
  \contact{\cSaboteur{}} That splendid author. You're looking forward to meeting him.
\end{contacts}


%%%%%
%% \starttag{<tag>} <elements> \endtag 
%% Valid <tag> values are blues, greens, abils, combat, mems, items,
%% whites, notebooks, cash, signs, ids.  These each correspond to a
%% type of macro defined in Lists/.
%%
%% By using \starttag, you can give this character <elements> of the
%% type corresponding to <tag>.
%%
%% Multiple uses of the same <tag> will simply add together.
\starttag{mems}
  \memfold{Open at the start of game.}{A series of explosions rock the ship, and you are thrown to your knees.}
  \memfold{Open if you see badge \#230}{It's Abigail!}
\endtag

\end{document}
