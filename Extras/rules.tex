\documentclass[sheet]{airship}
%% document-wide tweaks
\interlinepenalty10000
\setstretch{1}
\def\mytype{Basic Rules}
\lfoot{}\rfoot{}

\begin{document}
%% cover page
\thispagestyle{empty}
\parindent0pt
\parskip0pt

\begin{center}\LARGE\bf\begin{tabular}{|c|}
  \hline \gamename\\ Dec 2nd, 2017 \\ Basic Rules\\ \hline
\end{tabular}\end{center}

\vfill\vfill

The following are the rules for {\em\gamename}, a real-time,
real-space roleplaying game sponsored by the MIT Assassins' Guild.
You are responsible for knowing these rules.  Many of them are
nigh-impossible to enforce and rely upon the honor system.  Do not
cheat.  Do not abuse loopholes.  Play fair.  Be your own harshest
critic.

\vfill

The {\bf gamemasters} ({\bf GMs}) run the game. If you have any
problems or questions concerning the game, contact a GM. Rulings they
make are final.  They may violate the letter of the rules to preserve
the spirit.  The GMs promise to be as fair and reasonable as possible.
Neither they nor these rules are perfect.
\vfill

For this game, your GM is {\bf Acata Felton}. If we aren't in the same physical location (during packet hand out or during game), I can be reached over email at \emph{acata.felton@alumni.stanford.edu} or you can text me at \emph{650-690-5628}, or call and leave a message.

\vfill

This game is intended to be fun.  Getting into character, roleplaying,
being dramatic, and playing competitively can all increase the fun of
the game.  Do not take the game too seriously.  Even if you are
losing, keep a good attitude.  When the game is over, the real winners
are the players with the best stories.

\vfill

This is only a game.  Everyone involved should act with courtesy,
sportsmanship, patience, and taste.  The GMs may expel anyone they
believe to be violating the spirit of the rules or the game.  Emotions
may run high.  If you think things are crossing the line from game to
reality too much, or if you are just getting too stressed, calm down
and maybe take a break.  Stay in control.  Use common sense.  Always,
play safely, then play to have fun.

\vfill

This game is a work of fiction.  Although it may refer to things in
the real world, it does so only for the sake of the scenario.  It does
not represent the opinions of the GMs or the MIT Assassins' Guild.
These rules are modifications of those used in previous games.  This
game and all materials thereof are copyright \the\year\ by
\SORT{GM}{alpha<}{\parsealpha}{\EVERY{GM}{}}{\MYplayer, }and the MIT
Assassins' Guild.

\vfill\vfill

\begin{center}\bf
  BROUGHT TO YOU BY THE MIT ASSASSINS' GUILD
\end{center}

\vfill

%\clearpage
%% Table of Contents page
%\thispagestyle{empty}
%\tableofcontents
\clearpage
\setcounter{page}{1}
\parskip5pt

\section{Scenario}

Welcome aboard the {\it HMS Dauntless}, the first production model in a new line of luxury-class airships in Her Majesty's Royal Air Force.  This flight marks her maiden voyage from London, England to Paris, France on New Year's Eve, 1888.  As such, there are some notable personages onboard, as well as a scheduled Dedication of the ship at 10:30PM (game time).  The ship is to arrive in Paris precisely at midnight, to a grand firework display over the Parisian skyline. At game start, the ship has been in the air for just a few minutes\ldots{}

\subsection{Alternate History}

The advent of steam technology marked a significant divergence in the history of the world.  Ships are able to fly through the air, people who have lost limbs are able to get mechanical approximations of these limbs, and there are machines that can process information faster than any human can, all powered by steam technology.  It is these very wonders of steam technology that allow ships in the same class as the {\it HMS Dauntless} to exist.  The Steam War, fought between Britain and France for political reasons and the impetus for many of these technologies, ended about twenty years ago, with the British Empire victorious but bloodied. Dogfighting in steam-driven skyships was evidence of a new, modern kind of warfare. 

Queen Victoria II is on the throne, and strict moral and social standards are enforced. The House of Lords is balanced precariously between the Conservative and Labour Parties, the latter of which is quickly gaining strength from the rapid influx of workers into steam-powered industries. Across the Atlantic, the American Civil War ended at about the same time, leaving four countries sharing the North American continent: the United States of America (USA), the Confederate States of America (CSA), the Republic of Texas, and the British province of Canada.

It has been a somewhat unusually warm winter, and the weather over the Channel tonight is clear\ldots{}

\section{Getting Started}

\subsection{Character Packets}

Your character packet is a big manila envelope.  It contains your
role: who you are, what you're up to; everything about your part as a
{\bf player-character} ({\bf PC}) in the game.  Read all the contents
and generally keep them with you during the game.  If you are missing
something or find something which doesn't seem to belong to you, tell
one of the GMs.  Character packets are confidential.  Game materials
which cannot be given to other players are marked ``Not
Transferable,'' whereas things which can be given to others are marked
``Freely Transferable'' or ``Game Item.''

%% no character names on badges, yes to character descriptions
\paragraph{Name-Badge:} A name-badge with your player name, character
name, a brief description of your character's physical appearance, and
a {\bf badge number} on it.  Wearing your badge shows that you are in
the game; wear it visibly while you are playing.  It represents your
character's body in-game.  Badge numbers are not in-game information.
See the {\em Character Bodies} and {\em Badge Numbers} sections for
more details.

\paragraph{Character Sheet:} Your character sheet describes who you
are and what you are up to.  It contains a list of everything else
that should be in your character packet.  Do not show or read your
character sheet to other players.

%\paragraph{Bluesheets:} A bluesheet describes information common to
%members of a group.  When in conflict, character sheet information
%overrides bluesheet information.  Do not show or read a bluesheet to
%other players.

\paragraph{Greensheets:} A greensheet describes and expands abilities,
mechanics, or in-game knowledge.  Do not show or read a greensheet to
other players.

\paragraph{Stat Card:} Your stat card lists your statistics.  You
might not know what all of your stats mean.  Do not show your stats to
others.  The reverse side is a {\bf death report}; fill it out and
give it to the GMs when your character dies.

\paragraph{Ability Cards:} An ability card explains a special ability
your character has.  The front side describes the effects; show it to
players when you use the ability.  The reverse is the rules of use and
must not be shown to other players.

\paragraph{Memory/Event Packets:} A memory packet is an envelope or
stapled piece of paper with a {\bf trigger} which describes when to
open and read it.  If the trigger is a number, open the packet when
you see something with that number.  If it's a quoted phrase, open
when you hear or read it in-game.  If it's a symbol, open when
instructed.  Do not take game action based on an unopened trigger.  Do
not show or read a memory packet to other players.

\paragraph{Items:} In-game items may be transferred from character to
character, and should be marked as such.  See the {\em Items Etc.}
section for more details.

\paragraph{Scenario:} A scenario gives you general knowledge of the
game and its setting. The scenario for this game is at the front of this document.

\subsection{Reality and Game Reality}

There is a big difference between reality and game reality.  Players
must treat each other with courtesy and explain to each other what
their characters perceive in confusing situations; e.g.\ ``My
character's hands are covered in blood,'' an {\bf out-of-game}
statement.  Characters are under no such restrictions, and may do what
it takes to further their goals; e.g.\ ``Uh, hi Bob.  Just got back
from the butcher shop,'' an {\bf in-game} statement.

{\bf Metagaming} is inferring in-game knowledge that is inappropriate
for your character from out-of-game information.  Do your best to not
metagame and especially to prevent the risk of metagaming.  Be your
own harshest critic.

\paragraph{Halts:} A halt pauses game action.  To call one, say ``game
halt'' in a clear and audible voice; other players around a corner
should hear you, but you shouldn't scare some poor grad student.  End
a halt by saying ``three, two, one, resume.''  Call a halt for one of
only three reasons: because a rule instructs you to, for safety and
similar out-of-game issues, or to pause game and fetch a GM (which you
should avoid).

\paragraph{Not-Here:} You may go not-here by turning your name-badge
around so the ``I'm Not Here'' side is showing (or by removing your
badge entirely, if you are leaving game).  Putting a hand on your
head, visible from a distance, helps if you're near other players.  Go
not-here for one of only three reasons: because a rule instructs you
to, to leave game, or to fetch a GM while in a halt (which you should
avoid).

%% last two sentences not for closed-time/space game
When you are not-here, your character is not there.  Your character
cannot see, hear, or remember any game actions or information you (the
player) happen to encounter.  Avoid other characters, common game
areas, game signs, or any sort of game interaction.

\paragraph{Non-Players:} Use tact and common sense when dealing with
non-players ({\bf NPs}).  You are encouraged to spread the gospel of
real-time, real-space roleplaying; however, many NPs prefer to sleep,
study, or work undisturbed.

NPs may not knowingly affect the game.  They and their rooms may not
be used to hold items or information.  They may not help you kill.  Do
not use the presence of NPs to hide from rampaging mobs that want your
blood.

Avoid conspicuous or threatening game actions in front of NPs.
Shooting your friend outside of a classroom one minute before class
lets out is a bad idea, as is screaming bloody murder down a hallway.
If, despite your most valiant efforts, some NPs do get upset, call the
GMs who will help calm them down.

%% not for closed-space games
%%\paragraph{Player Rooms:} Players may retreat to their rooms to study,
%%sleep, or whatever in safety.  Your character may not enter a player's
%%room unless invited in-game.  This has traditionally been called the
%%``jhereg rule.''  Do not use your room as an impenetrable meeting
%%place or stash site.  If your character is in-game in your room, other
%%characters may interact with (kill, torture) you.  Roommates and
%%similar are considered to have separate rooms for this rule.

\paragraph{Observers:} An observer is someone not playing the game who
has agreed to watch.  They generally wear an observer headband or an
observer name-badge.  Observers have traditionally been called
``ghosts.''  They should stay out of the way; you can always ask an
observer to leave.  If a friend who is not playing wants to observe
game, send them to the GMs.

\paragraph{Non-Player-Characters:} Non-player-characters ({\bf NPCs})
are characters in the game's universe not played by a full-time
player.  They are minor characters, bit parts, or random people.  Some
may have name-badges; sometimes called ``GM plants,'' these are often
not readily distinguishable from PCs.

\paragraph{Mechanics:} Many actions your character can take, such as
walking, talking, and general interaction with other characters, are
represented by you doing them.  Others, like combat, are performed via
abstract mechanics, which are described in ability cards, greensheets,
and rules.  The abstract information for mechanics (like badge
numbers) may not be discussed in-game.  If you want to do something
special for which there is no mechanic, ask a GM.

Become familiar with your mechanics before game starts, especially
those which occur under time-pressure (like combat).  Game action will
not stop for memory packets, greensheets, or such.

A {\bf kludge} (and derivative forms like ``kludge-ite'') is something
impervious to logic and cleverness, usually for game-balance.  You
can't affect a kludge without a specified mechanic.

{\bf Zone of Control} ({\bf ZoC}) is a rough distance measurement.
You are within ZoC of someone if your outstretched fingers can touch
their outstretched fingers.  Double-ZoC is twice this distance,
triple-ZoC is three times, etc.

{\bf Headbands} represent obvious visual effects; wear them visibly on
your head.  If you see a headband and don't know what it represents,
ask.  If you are wearing a headband, tell people what their characters
see.

%To play {\bf Rock, Paper, Scissors} ({\bf RPS}), you and your
%opponent(s) say ``one, two, three, show'' in unison.  On ``show''
%everyone displays and compares their chosen symbol.  Rock is a closed
%fist.  Paper is a flat hand with palm down.  Scissors is a fist with
%the first two fingers extended, looking vaguely like a pair of
%scissors.  Rock defeats (crushes) scissors, scissors defeats (cuts)
%paper, paper defeats (covers) rock, and any symbol ties with itself.
%You may see or be able to play other, special symbols; the wielder
%will know what happens.

\paragraph{Safety:} This is a game.  Real violence is unacceptable.
Game action should cause no real-world damage, either to people or
property.  If something dangerous is happening, call a halt.  Stay in
control, use common sense, and do not endanger yourself or others.
You should not run or otherwise force your way into or through someone
else's ZoC, and you should not make physical contact with another
player without permission.

\subsection{Basic Strategy}

Make sure you understand the rules.  If you are completely confused,
get a GM who will try to help you out.  Make sure you know enough
about your character to role-play him or her when you start talking to
other people.  Read through your entire packet a couple of times, and
skim through it again right before game starts.  If you don't know
something about your character, ask a GM.

As a character, your first priority should be to open lines of
communication.  Contact people, show up at meetings, and chat.  Try to
be easy to get in touch with.  Ask people questions on relevant
subjects.  They'll probably lie, but you may find something out.

There are no guarantees that you can trust anyone, but since
cooperation is the key to accomplishing things, you will be forced to
trust people anyway.  The most trustworthy people are probably those
who need you.

\section{Items Etc.}

Many in-game items are represented by little white cards with a number
and description.  Item cards may be shown to others, passed around,
stolen, etc.  The {\bf item number} on the card is not in-game
information and may not be discussed.  Not all in-game items have
cards or numbers; whatever they are represented by should be clearly
marked ``in-game item'' or ``freely transferable.''

Use common sense.  You can't carry a hundred rocks in your pocket,
fold a sword in half, or hide a life-sized statue in a fire hose.  You
can't stop a bullet with a set of blueprints or rip apart a metal safe
with your bare hands.  Even if your bag can carry a shovel in it, the
shovel noticeably sticks out (``you see a shovel sticking out of my
bag'').

\paragraph{Written Information:} If you write in-game information down
on a piece of paper, that paper is now an in-game item and must be
clearly marked as such.  Don't write in-game information on
out-of-game documents (character sheet, etc.).  Don't write
out-of-game information (like memory packet triggers) on in-game
documents.

\paragraph{Envelopes:} Some items and locations may have an attached
envelope (or just be a labeled packet or folded paper).  The envelope
may include directions for when to open these (``open packet if you
press the big red button'' or ``open packet if you eat this'');
otherwise you may only open them if instructed.  Close them when you
are done.  Open and close packets gently.

\paragraph{Signs:} Some locations and other game materials are
represented by signs or packets posted throughout game area.  You may
read any signs and must follow any rules printed on them.  If a sign
or packet doesn't have some sort of in-game description (it only has
out-of-game mechanics information, like a number or just a colored
dot), then your character doesn't even see it or know that anything
unusual is there.  If you see a sign on a closed door, you must read
the sign before you open the door.

\paragraph{Bulkiness:} A bulky item is too big or heavy to be carried
or concealed freely.  Bulkiness is measured in {\bf hands} or {\bf
dots} (how many hands it takes to carry it).  If you are carrying a
bulky item, make it clear to onlookers (hold the card).  A hand
carrying a bulky object may do nothing else.  With one hand less than
required, you may drag a bulky item at a slow pace.

\paragraph{Props:} Some items may have props (physical representations
or {\bf physreps}) associated with them.  The card and physrep should
be kept together.  If they are separated, the card is the real item.
Prop items are as bulky as the physrep.  They can be carried in bags
that can hold them, on straps that are attached to them, etc.

\paragraph{Containers:} Some items, like crates or personal bags, have
a {\bf capacity}.  Capacity is measured in dots or hands; this is how
many dots of items can be stored within.  You can put as many
non-bulky items inside as is within reason.  A container may have a
capacity bigger than its bulkiness; use common sense when nesting
containers.  Put contained item cards inside the envelope attached to
the container card.

\paragraph{Character Bodies:} A body is a bulky item and is 
represented by a name-badge.  It must be willing or unable to resist
for you to carry it.  Carry the badge conspicuously.  Onlookers can't
tell if it's dead without close examination, unless it would be
obvious (like headless). Bodies are typically two or three hands bulky.

\paragraph{Unstashable Items:} Unstashable items can't be hidden or
left behind.  They look too important, valuable, or interesting; NPCs
will not let them stay there.  This is a kludge.  If you're not
leaving an unstashable item in another PC's care, and you want to
leave it behind, give it to a GM or observer.  You may leave it in
plain sight in a public area if there are other PCs around.

\subsection{Searching, Stashing, and Stealing}

\paragraph{Places:} To search a place, search it.  Normal items can be
stashed in any reasonable, legal place.  Don't put items behind locked
doors, or inside ceilings; consequently, don't go rummaging through such places for
game items.  Don't stash or search in places that are not in-game; see
the {\em Game Areas} section for more information.

\paragraph{People:} All searches of characters or their belongings are
conducted via player dialogue.  Someone must be willing or unable to
resist for you to search them.  You need at least one free hand to
search someone.  Anyone within ZoC of either you or your victim can
prevent the search by saying ``I stop you'' or an equivalent phrase.

You can perform a {\bf pat-down search}, which will only reveal the
presence of weapons.  This takes as much time as it takes your victim
to tell you what you find.  If you're the victim, do this at a
reasonable pace.

A {\bf total search} is an invasive, complete search of a character's
clothing.  This reveals all in-game items, and takes as long as your
victim spends handing over possessions.  If you're the victim, hand
over items at a reasonable pace.

\paragraph{Bags:} To search a bag in someone's possession, say ``I
search your bag.''  This proceeds just as a total search.

To search a bag that is obviously in-game (has an attached, displayed
item card), search the physrep.  Item cards in the bag must be in
reasonable places.  If the item card has a capacity and an envelope,
the bag is just a prop and all in-game items should be in the packet.

To search a bag that is not obviously in-game (no visible item card
attached), spend thirty seconds by the bag, put a ``searched (see a
GM)'' note on the bag, and come tell a GM.  Tell any onlookers that
they see you searching through the bag.  Search an attended bag via
player dialog with the owner; they must be willing or unable to
resist.  If someone searches a bag you are holding, hand over all game
items inside at a reasonable pace.

If you find a ``searched'' note on your bag, come see a GM.  To
declare a bag out-of-game, label it ``no game items.''

%% replaces entire above \paragraph
%\paragraph{Bags:} To search an in-game bag (has an attached, displayed
%item card), search the physrep.  Item cards in the bag must be in
%reasonable places.  If the item card has a capacity and an attached
%envelope, the bag is just a prop and all in-game items should be in
%the packet.
%
%Note: all in-game bags must either have an attached container envelope
%or be fully in-game.

\section{Violence, Damage, and Death}

\subsection{Health States}

Characters have five possible states, concerning health and damage.
When you are {\bf fine}, you may act freely.  When you are {\bf
restrained}, you are helpless and may do nothing but talk.  When you are
{\bf knocked out}, you will wake up in five minutes.  When you are
{\bf wounded}, you are unconscious, bleeding, and will die in five
minutes.  When {\bf dead}, you are dead.

When knocked out or wounded, fall down and drop anything you are
holding.  Just lie there.  You won't be doing much of anything until
you wake up.  Do not listen to conversations going on.

Dead men tell no tales.  If dead, do not give out any information
about your character or death to any players.  You may remain on the
scene to play the part of your corpse; describe obvious information to
onlookers (``I have a gunshot wound in my back'').  When you leave,
place the front of your name-badge with a description of the body's
obvious state.  Take the ``I'm Not Here'' side to wear.  Stack your
items with your body.  Fill out your Death Report.  Make sure the GMs
know about your death.  If your death becomes generally known to the
other characters, you may be able to become an observer.  Until the
game is over, you may not convey game information to any player.

\subsection{Weapons}

All weapons have both a physrep and an item card; keep these together.
Weapon effects are on the card.  To use a weapon, you must have it in
your hand and unobstructed.  Display it in an obvious manner.  You
cannot hold more than one weapon in a hand.  You may only use one
melee weapon at a time.

\subsection{Killing Blow}

A killing blow will kill a helpless victim.  Your victim must be
within ZoC and either unconscious or restrained.  You must use a
weapon (melee or ranged).  Clearly incant ``killing blow one, killing
blow two, killing blow three'' at a reasonable pace.  During the
incant, if you are attacked or if someone within ZoC says ``I stop
you'' or an equivalent phrase, you are stopped.  To stop a killing
blow, either attack the person doing it or say ``I stop you'' within
ZoC.

\subsection{Ranged Combat}

Ranged combat is real-time and mostly based upon player skill at
firing and dodging physreps.  Keep it safe.  Hits to anywhere on the
body count the same; don't aim for the head.  If a projectile hits
clothing or long hair such that it would not hit the body when passing
through, it doesn't count.  Hits to an item you hold count as a hits
on you, not the item.  If there is a conflict over whether or not a
projectile hit, the shooter calls the shot.

All ranged projectiles have the same effect: if and when you are hit
by one, you become wounded.\footnote{``Fall down and start bleeding.''}  Bullets are represented by nerf darts.  Ammunition is limited; ammo that hits the floor leaves game. Each gun in game will have specific instructions on it for how to reload it, if this is possible.


%
%Arming or disarming a bomb requires an appropriate ability card.  If a
%bomb explodes, it will be made obvious by a halt being called.  If you
%are within arm's reach of a bomb when it explodes, you are dead.  A
%bomb will have a piece of string attached to it.  If, when stretched
%out (even around corners), the string can touch you, you are wounded.
%Once the dead and wounded have been determined, game will resume.

\subsection{Martial Combat}

%% intro
All characters have a {\bf Combat Rating} ({\bf CR}) stat.  This
represents your basic skill in martial combat; you use the same number
for attacking and defending.  Someone with a CR of one can't fight
very well.  Someone with a CR of three is somewhat burly or skilled.
When using this stat, you may pull your punches by using a lower
number.
  
%% offense
To martial-attack someone, clearly state your attack and CR
(``\aKnockOut{} 2'', ``\aWound{} 2'', etc.) from within ZoC.  You need
the ability card for any attack you make; you don't have to display
it.  Your attack must resolve before you make another; otherwise, you
may act freely.  If an ally directs {\bf \aAssist{}} at you after you
attack, you may, within 2 seconds, restate your attack with the
\aAssist{}'s CR added (``\aWound{} 3'', ``\aAssist{} 2'', ``\aWound{}
5'').  \aAssist{} does not change your CR for defense.  You may ignore
an \aAssist{}.

%% defense
When martial-attacked, resolve by comparing the attack against your
CR.  If your CR is lower, take the effects; else, say ``{\bf resist}''
and the attack has no effect.  If you neither say ``resist'' nor state
your own attack within two seconds of the incant's end, you are
surprised and the attack just works.  The attack begins when the
incant begins; until you resolve, all of your actions other than
martial attacks are interrupted; serial attacks don't prevent simple
actions (talking, weapon-drawing, ranged attacks) in-between.  Resolve
all attacks alone, in the order they occur; choose the order if it is
unclear.  If you are attacked with ``{\bf waylay}'' instead of a CR
(``\aKnockOut{} waylay''), the attack just works.

\paragraph{Martial Attack Abilities:} Here is a list of attack
abilities.  Everyone has \aKnockOut{}, \aWound{}, and \aAssist{}.
Only some people will have \aDisarm{} and \aRestrain{}.  Other attack
abilities may exist.\nopagebreak

\begingroup
  \MAP{Abil}{%
    \setbox0\hbox{\phantom{w}{\em Effect}: \MYeffect}%
    \par{\bf\MYname}: \MYtext\hfill\null\hskip\wd0\null%
    \hskip-\wd0 plus1fill\box0%
    \nopagebreak\par%
    }
  \aKnockOut{}
  \aWound{}
  \aAssist{}
  \aDisarm{}
  \aRestrain{}
\endgroup

\subsection{Stealth}

Stealth abilities represent sneaking up on a victim with obvious
intent to invade their personal space, probably to attack them by
surprise or to pick their pocket.

To use a stealth ability, you must be within ZoC of your victim.  Form
the sign of the devil (index and pinky fingers extended, thumb holding
other two fingers down) and extend it along the direct, unobstructed
line from your shoulder to the victim's head.  Hold this position for
the time specified by your ability.  Before this time is up, the
ability is thwarted if anyone attacks you or if the victim notices the
symbol.  If they react in any way to the symbol, they have noticed;
you (the attacker) make the call.

If you notice someone using a stealth ability on you, make it obvious.
``I notice you'' is unambiguous; use it if you can.  Once a stealth
ability is finished, you may not retroactively have noticed.

\paragraph{Waylay:} You can attack by surprise as a stealth ability.
You must hold the symbol for five seconds.  If you succeed, you may
replace your CR with ``waylay'' for a single immediate attack on your
victim.

\section{Miscellaneous}

\paragraph{Headband Colors:} A white or yellow headband represents an
observer.  A red headband is a fearsome, rampaging monster. Other colors represent something that looks dangerous or at
least unnerving.  If you see someone wearing another color of
headband, it might be best to ask what it represents from a safe
distance.

\paragraph{Badge Numbers:} The first digit of your badge number is
your character's apparent age in decades. The second digit of your badge number is how many hands worth of bulkiness your body is.
%  The second digit is your
%character's apparent burliness: a ``3'' is pretty skinny, a ``5'' is
%average, and an ``8'' is huge and muscular.

\paragraph{Explosives:} There may be items in game that can be used as explosives. They may not be moved; this is a kludge. The effects of these items will be described on the item.  Often, the only way to know if an explosive is armed and if it can hurt you is to come up to it and examine its item card. Explosives will typically be mechaniced by a GM on the scene.

\paragraph{Rope:} Rope is freely available.  Make an item card for it.
To tie someone up, they must be either willing or helpless.  If you
get tied up with rope, you become restrained.  If you are conscious
and left alone, you can wriggle free in five minutes.  If it ever matters, rope is a one-hand bulky item.

\paragraph{Game Times:} Dec 2nd, 2017 will have two simultaneous runs of game on Saturday, from 2pm-6pm. Be sure
to arrive to room {\bf 210} in the Education building by {\bf 1:30 pm} to get settled.Feel free to arrive up to an hour early to read over your character packet one last time, cut out your abilities, etc. Surviving PCs are expected to be in-game for the 
entirety of game.  In the event of some
emergency that causes you to need to leave early, please tell a GM
before you go.  There is some chance the game may end early.  If this
happens, the GMs will let you know.  Cleanup and Wrapup in room 210 lobby will
immediately follow the end of game.

\paragraph{Game Areas:} Game will take place on the second floor of the Education building. We may or may not have additonal rooms on the first and third floor. An announcement will be made before game, explaining which parts of the building are part of which run. The edges of game will be clearly marked.  All publicly-accessible areas are
considered in-game (your character can move about freely in them).  As
usual, avoid places it is illegal for you to go, or not all players have access like areas under
construction, private offices, etc.

%%When in living areas, such as dorms, remember the {\em Player Rooms}
%%section.  Many living areas on campus are not technically accessible
%%to all players.  Whether or not to take game action in your living
%%area is left to player judgment.

%%There are some areas on campus that are not publicly in-game.  You may
%%not enter them in-character unless explicitly instructed to; if you
%%happen to be in them your character is not there.  These areas are:


\paragraph{Doors and Locks:} Some doors or items in game are {\em locked}. You may not open them or get past them unless you have a key with that item's number on it, or have some other method of opening locks. Closing such an item or door locks it again. This is a kludge for game balance.

%\clearpage
\section{Closing Notes}

These rules are imperfect.  The GMs may violate the letter of the
rules to preserve the spirit.  We hope these rules are reasonably
clear, but if you have any doubts about your interpretation, talk it
over with us in advance.  We should also add, as much as we hate to
admit it, we GMs are human: when all of our carefully laid plans are
going haywire, we may lose our cool.  The best way to deal with people
is remaining calm and friendly, especially when everyone is tired and
hungry.

We hope you have lots of fun.  Good luck.

\end{document}
