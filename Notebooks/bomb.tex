%%%%%
%%
%% Research Notebooks live in this directory.  This file doubles as a
%% latex'able example notebook.
%%
%% Notebook macros (in ../Lists/notebook-LIST.tex, presumably) each
%% have a file that lives here.  The argument to \startnotebook{...}
%% probably should be the macro for the given whitesheet.  However,
%% you can also just use \name{Some Text} if you want.
%%
%% Note that every \startnotebook command needs a matching
%% \endnotebook command.  Also note that no ownership information
%% appears on the notebook.
%%
%%%%%

\documentclass[notebook]{airship}
\begin{document}

\startnotebook{\nBomb{}}



\begin{page}{Begin}

This notebook is not transferable. If you want to attempt to defuse the bomb, spend 1 minute with a {\bf Toolkit} and proceed to the next page.
\end{page}



\begin{page}{Tick-tock.}
{\it As you remove a panel on the device, you trip an internal switch. The device beings to beep.}

Start the nearby timer. Proceed to page \nbref{A}.

\end{page}



\begin{page}{B}

{\it The second step is to ground out the watchdog circuit.}

\begin{enum}
\item Shuffle the nearby deck of cards 3 times. 
\item Deal out 5 cards.
\item Deal out 4 cards below those.
\item You may discard one of the top row of cards and replace it with one of the bottom row. Discard the bottom row.
\item Repeat 3-4 until the top row has four cards all of the same value. If you have a {\bf Soldering Iron} and a {\bf Metal Sheet}, you may stop when the top row has three cards with the same value. Destroy the {\bf Metal Sheet}. Feel free to shout for items. If you run out of cards, begin again from step 1. 
\item Proceed to page \nbref{C}.
\end{enum}

\end{page}




\begin{page}{A}

{\it The first step is to strip the external housing of the bomb.}

\begin{enum}
\item Shuffle the nearby deck of cards 3 times. 
\item Deal out 5 cards.
\item Deal out 4 cards below those.
\item You may discard one of the top row of cards and replace it with one of the bottom row. Discard the bottom row.
\item Repeat 3-4 until the top row has cards all of the same suit. If you have {\bf Copper Wire}, you may stop when the top row has two cards with the same value. Destroy the {\bf Copper Wire}. Feel free to shout for items. If you run out of cards, begin again from 1. 
\item Proceed to page \nbref{B}.
\end{enum}
 
\end{page}



\begin{page}{C}
{\it Now, cut the red wire! Wait, maybe the blue one?}

\begin{enum}
\item Shuffle the nearby deck of cards 3 times. 
\item Deal out 5 cards.
\item Deal out 4 cards below those.
\item You may discard one of the top row of cards and replace it with one of the bottom row. Discard the bottom row.
\item Repeat 3-4 until the top row is a straight flush (numbers in sequence, all the same suit). If you have a {\bf Rigging Kit}, you may stop when the top row has two pairs in it. Feel free to shout for items. If you run out of cards, begin again from step 1. 
\item Proceed to page \nbref{D}.
\end{enum}


\end{page}



\begin{page}{D}

{\it The beeping stops.}

Success! Stop the timer. Write ``Defused'' on the sign. 

\end{page}

\endnotebook

\end{document}
