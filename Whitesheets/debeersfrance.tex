%%%%%
%%
%% Whitesheets live in this directory.  This file doubles as a
%% latex'able example whitesheet.
%%
%% Whitesheet macros (in ../Lists/white-LIST.tex, presumably) each
%% have a file that lives here.  The argument to \name{...} probably
%% should be the macro for the given whitesheet.  However, you can also
%% just use \name{Some Text} if you want.
%%
%% Note that for whitesheets, \name doesn't actualy typeset anything.
%% If you want the ``title'' of the IG document to appear, typeset it
%% how you want it.  Similarly, no ownership information appears on
%% the sheet.
%%
%%%%%

\documentclass[white]{airship}
\begin{document}

\name{\wDeBeersFrance{}} %% used as a label, doesn't typeset anything
{\em (This is a letter cleanly inscribed on sturdy papyrus. There is no return address.)}

December 28th, 1888

You don't yet know me, but I've definitely heard of you, Mr. Rousseau.

My name is Cecil John Rhodes. I'm writing from the Dutch colony at the Cape, where I have been a businessman and statesman for the last decade, after my education at Oxford. My associate Charles Dunell Rudd and I have discovered a way to make the heretofore unprofitable and impotent diamond mining operations in South Africa worthwhile with an ingenious new way of extracting diamonds from rock. We plan to consolidate some claims of ours around a farm in one of the western provinces of Cape Colony, and launch the De Beers Diamond Company.

What we sorely lack, at the moment, is capital and protection from a major international power, and I hope that you and the French with may be able to provide such a thing. To show that I am in earnest, I have given Lord Blackwell a package of diamonds and a third of the new Company's shares to hide aboard the ship, and have instructed him to negotiate a price with you and and any other competitors. The one that can promise me the capital we require, and the security needed for such an operation may have the shares, which will soon be worth much more than the accompanying gems, if we succeed in our plans.

Speak with Lord Blackwell at your earliest convenience.

Faithfully Yours, 

Cecil John Rhodes, Esq.
Lord Mayor of the Cape Colony

\end{document}
